\section{Minimum Viable Product Approach}

In the development of the GenAI Advisor, a structured Minimum Viable Product (MVP) approach was followed to ensure systematic progress while respecting the constraints of time, computational resources, and project evaluation needs. The MVP methodology allowed for the early identification of potential technical limitations while enabling the iterative refinement of core functionalities essential to the project.

The MVP was designed to operate fully offline, respecting user data privacy and ensuring explainability. It integrates four layers: data ingestion, strategy engine, explanation generation, and a user-facing Streamlit interface. The initial focus was on constructing a thin, end-to-end pipeline capable of performing equity data ingestion using Yahoo Finance, caching the data locally in CSV format to avoid repeated API calls while supporting reproducibility.

The strategy engine implemented a simple rule-based mechanism using a moving average crossover technique, providing structured recommendations (BUY or HOLD) alongside clear reasons based on the underlying signal. This approach balanced simplicity with interpretability, aligning with the goal of transparent, explainable AI systems suitable for retail investors.

Explanation generation was initially prototyped with placeholder explanations before being integrated with a locally hosted Mistral 8B model via Ollama, enabling offline large language model inference. This allowed for the generation of user-friendly, clear explanations for investment signals without requiring online API dependencies.

A Streamlit frontend was developed to facilitate user interaction, enabling the entry of tickers, the execution of the pipeline, and the clear display of structured recommendations, explanations, and recent price charts. This immediate feedback loop supported the testing and evaluation processes, aligning with agile development practices.

Pytest was configured to establish a lightweight test-driven development environment, ensuring that data ingestion, strategy engine outputs, and explanation generation remained reliable as further functionalities were layered onto the system.

The MVP was then extended with a backtesting utility, enabling the analysis of recommendations against historical price movements over a defined lookahead period. A batch backtesting pipeline was constructed to systematically test multiple tickers across various dates, generating CSV outputs to support quantitative evaluation. An evaluation notebook was prepared to visualise the distribution of price movements following recommendations, allowing for the analysis of the effectiveness of the generated signals.

This MVP approach ensured that a functioning, test-backed advisor system was available early in the project timeline, providing a robust foundation for evaluation and report writing, and allowing for targeted refinement and iteration in subsequent development sprints.

\section{Implementation Challenges in the MVP}

During the MVP development, several non-trivial challenges arose that required careful technical decisions to ensure system robustness and correctness. Firstly, handling timezone consistency was critical when implementing the backtesting pipeline. The \texttt{yfinance} library returns time series data indexed with a timezone, whereas initial cutoff dates for historical evaluations were timezone-naive, resulting in type errors during comparisons. This was resolved by explicitly localising the cutoff dates using \texttt{pd.to\_datetime().tz\_localize('America/New\_York')}, ensuring consistent and error-free slicing of historical data.

Another challenge involved achieving reproducible and API-resilient data ingestion through CSV caching. Early iterations faced hidden header misalignments and unnamed indexes when saving and reloading data, which caused failures in downstream modules. This was addressed by enforcing consistent use of the \texttt{index\_label} parameter during saving and explicitly specifying \texttt{index\_col} during loading, ensuring data could be reliably reused without corruption.

Integrating a local large language model explanation generator with \texttt{Ollama} also required precise engineering. Instead of using an external API, a local Mistral 8B model was leveraged to maintain offline functionality and data privacy. This necessitated the design of a robust subprocess management approach, correctly piping structured prompts and capturing outputs while handling encoding and potential model unavailability gracefully.

Additionally, implementing consistent test discovery using \texttt{Pytest} required an understanding of Python's module resolution, as the project initially failed to locate modules due to the package structure. This was resolved by enforcing the use of \texttt{PYTHONPATH=.} during test execution, ensuring reliable test discovery across environments.

Finally, careful attention was required in the design of the backtesting pipeline to prevent future data leakage while computing lookahead returns. Ensuring that data used to generate recommendations strictly respected the cutoff date, and accurately measuring post-recommendation price movements, was critical for the validity of the evaluation metrics used in the later stages of the project.

Collectively, addressing these challenges not only ensured the stability and correctness of the GenAI Advisor MVP but also demonstrated disciplined software engineering practices that align with industry and academic expectations for reliable, reproducible systems.

\section{Expansion of the MVP with Advanced Signal Architecture}

Following the initial MVP construction, the GenAI Advisor was extended with a modular, advanced strategy engine architecture during the first development sprint. This refactor transitioned the single-signal SMA crossover logic into a structured multi-signal system, allowing for the integration of additional indicators including the Relative Strength Index (RSI) and the Moving Average Convergence Divergence (MACD).

Each signal was implemented as an independent module within the system, supporting clean separation of concerns and enabling systematic unit testing. The RSI module detects oversold conditions, while the MACD module identifies momentum changes based on the convergence and divergence of moving averages. These signals were integrated into a combined recommendation engine that uses a voting logic approach: the system recommends a BUY signal if any of the signals indicate favourable conditions, while defaulting to HOLD otherwise. This design allows the system to leverage multiple perspectives on market conditions while maintaining transparency and interpretability.

The modular structure not only facilitated the immediate extension of the strategy engine during this sprint but also positioned the system for future enhancement with additional signals or weighting schemes. Throughout this refactor, test-driven development was maintained using Pytest, ensuring the correctness of individual signals and the consistency of the combined recommendation logic.

This enhancement significantly increased the system's analytical capabilities while preserving the key goals of offline operation and explainability, aligning with the broader objectives of the GenAI Advisor project.

\subsection{Strategy Engine Enhancements}

Building upon the Minimum Viable Product developed in Week 1, the strategy engine was expanded to include additional technical indicators to enhance the quality and diversity of signals while maintaining transparency and interpretability. 

Two new signals were implemented:

\begin{itemize}
    \item \textbf{Bollinger Bands}: A volatility-based indicator using a 20-day simple moving average (SMA) with upper and lower bands set at two standard deviations. A BUY signal is generated when the closing price crosses below the lower band, indicating potential oversold conditions and a likelihood of reversal.
    \item \textbf{Stochastic Oscillator}: A momentum indicator utilising the \%K and \%D lines over a 14-day window. A BUY signal is triggered when the \%K line crosses above the \%D line while both are below the threshold of 20, signalling oversold conditions with potential upward momentum.
\end{itemize}

These indicators complement the existing SMA Crossover, RSI, MACD, and the ML-based classifier, creating a hybrid strategy engine capable of generating diverse, explainable signals across trend, momentum, and volatility dimensions.

The implementation of these signals was modular, with each signal encapsulated in a dedicated module (\texttt{bollinger.py}, \texttt{stochastic.py}) within the \texttt{strategy\_engine} folder, following the same structured dictionary return pattern as the existing signals. This consistency ensured seamless integration with the central aggregation function in \texttt{engine.py}, enabling the new signals to contribute to the hybrid voting mechanism used for recommendation generation.

To maintain robustness, unit tests were created to verify that both the Bollinger Bands and Stochastic Oscillator modules returned valid recommendations (BUY or HOLD) and contained structured, interpretable outputs. These tests ensured reliability and integration readiness within the broader offline pipeline.

By expanding the strategy engine with these additional signals, the system now offers a richer set of signals while preserving explainability and reproducibility, aligning with the goals of the GenAI Advisor project to provide transparent, offline investment insights to retail investors.
