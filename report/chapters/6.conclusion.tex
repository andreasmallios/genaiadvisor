\section{Conclusion}

\subsection{Summary}
The implemented system constitutes a functional Minimum Viable Product (MVP) that integrates data ingestion, technical signal computation, ensemble logic, and explainable recommendation generation through a cohesive interface. This modular architecture facilitated rapid prototyping, iterative evaluation, and extensibility across system components. The ability to trace recommendations back to interpretable signals and test outputs in a reproducible manner established a strong foundation for further development.

\subsection{Future work}
Future work is centred around improving the system’s expressiveness, adaptivity, and user experience:

\begin{itemize}
\item \textbf{Migration to SQL:} Although CSV files were used for data storage in this implementation due to their simplicity and transparency, migrating to a structured relational database like MariaDB would be a natural evolution. This would improve scalability, allow for more complex queries, and enable multi-user support. Given the educational and offline-first objectives of this version, database integration was deferred and is proposed as future work.
\item \textbf{User interface improvements:} While the Streamlit interface effectively demonstrates system functionality, future UI enhancements will focus on interactivity and visual storytelling. This includes time-series overlays for each signal, interactive sliders to simulate portfolio scenarios, and user-driven feedback loops. Richer visualisations will also support educational transparency, a core design objective of the tool.
\item \textbf{Explanation pipeline enhancements:} Current prompts to the local language model are stateless. Incorporating context-aware memory—such as dialogue history, prior user queries, or preference metadata—would allow for more personalised and coherent multi-turn explanations. This may involve integrating a lightweight memory layer or using conversation-aware prompt templates to simulate state retention.
\item \textbf{Empirical signal weighting:} The existing strategy engine applies fixed weights to each indicator. However, not all signals contribute equally to predictive accuracy across assets or timeframes. Future iterations will incorporate a data-driven signal weighting mechanism, potentially via logistic regression or Bayesian optimisation, based on historical backtest performance. This would enable adaptive ensemble construction and improved recommendation fidelity.
\end{itemize}

Collectively, these planned developments aim to deepen the system’s interpretability, robustness, and real-world decision support potential—particularly for novice retail investors seeking algorithmic guidance that is both transparent and context-sensitive.